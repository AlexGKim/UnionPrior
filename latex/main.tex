\documentclass[11pt, oneside]{article}   	% use "amsart" instead of "article" for AMSLaTeX format
\usepackage{geometry}                		% See geometry.pdf to learn the layout options. There are lots.
\geometry{letterpaper}                   		% ... or a4paper or a5paper or ... 
%\geometry{landscape}                		% Activate for rotated page geometry
%\usepackage[parfill]{parskip}    		% Activate to begin paragraphs with an empty line rather than an indent
\usepackage{graphicx}				% Use pdf, png, jpg, or eps§ with pdflatex; use eps in DVI mode
								% TeX will automatically convert eps --> pdf in pdflatex		
\usepackage{amssymb, amsmath}

%SetFonts

%SetFonts


\title{Unity ``Binned'' Prior}
\author{Alex Kim}
%\date{}							% Activate to display a given date or no date

\begin{document}
\maketitle

The DESI plus SN results have folks in a flurry.  Talking with DESI and David it is clear that DESI is incorporating Union3
into their analyses incorrectly albeit in a subtle way;  it is of interest to explore how misapplication
of Union3 could affect  DESI $w_0$--$w_a$  fits.

In the Unity ``Binned'' model, the distance modulus model function
is the sum of the distance modulus for $\Omega_M=0.3$  $\Lambda$CDM plus a second-order spline specified by node values
at a fixed set of redshifts.  The prior on the node values are taken to be standard normal distributions.  The results of the fit are provided by David as
$\mu_{\Lambda \text{CDM}}(z;\Omega_M=0.3) + n$ and the  Hessian of the $n$ posterior,
where $z$ are the fixed redshifts and $n$ the node-value parameters.
DESI takes the Union ``Binned"-model parameter posterior output as data, interpret the covariance matrix as a Gaussian uncertainty,
and include them in a larger likelihood. 

DESI uses the posterior to fit a $w_0$--$w_a$ cosmology model, which is not embedded in the Unity ``Binned'' model.  Nevertheless,
the ``Binned'' model can come close by fixing the values at the modes to match those of the $w_0$--$w_a$  model and swallowing the
inconsistencies between the physics and spline predictions between the nodes.  For the purposes of this work, I consider
as an alternative  ``$w_0$--$w_a$'' model  the subspace of node values that give
$\mu$ values accommodated by the actual $w_0$--$w_a$
model.  The transformation between the subspace parameters $\theta=\{\Omega_M, w_0, w_a\}$ and their corresponding
node-value $n$ parameters is given by
\begin{align}
	n &= f(\Omega_M, w_0, w_a; z) \\
	& \equiv \mu_{w_0 w_a \text{CDM}}(z;\Omega_M, w_0, w_a)  - \mu_{\Lambda \text{CDM}}(z;\Omega_M=0.3).
\end{align}

The standard normal prior on the nodes corresponds to a prior on the  $w_0$--$w_a$ manifold
\begin{equation}
p_U(\Omega_M, w_0,w_a) = \mathcal{N}(f(\Omega_M, w_0, w_a; z),1)  \sqrt{\det{\left(J^T J\right)}},
\end{equation}
where $J$ is the Jacobian matrix of $f$, $J^TJ$ is the Gram matrix.
Surfaces of $\ln{p_U}(\Omega_M, w_0,w_a)$ values\ for a grid of values of $\Omega_M$
are plotted in Figure~\ref{fig:priors}.
The Union3 posterior, denoted as $p_U(n)$, has this prior baked in.  For a flat prior in $\Omega_M$--$w_0$--$w_a$
the new posterior on the manifold can be determined from the original posterior as
\begin{equation}
p_F \propto \frac{p_U(n)}{\mathcal{N}(n,1)  \sqrt{\det{\left(J^T J\right)}}}
\end{equation}
Inspection of Figure~\ref{fig:priors} show that there will be a shift in the contours toward lower $w_0$ higher $w_a$.

\begin{figure}[htbp] %  figure placement: here, top, bottom, or page
   \centering
   \includegraphics[width=5.5in]{../result.png} 
   \caption{Surfaces of $\ln{p_U}(\Omega_M, w_0,w_a)$  in $w_0$--$w_a$ for a grid of values of $\Omega_M$.    The star (in the upper-left plot) shows the location of the maximum.
   Plotted as points are the best fit  DESI+Union3
   result and  $\Lambda$CDM.}
   \label{fig:priors}
\end{figure}

The Union3 posterior values $p_U$ (as approximated by the Heissian)  on a grid of $\Omega_M$--$w_0$--$w_a$ values
are shown as the red contours in Figure~\ref{fig:posterior}.   The posterior $p_F$ assuming a flat
prior in $\Omega_M$--$w_0$--$w_a$  are overplotted in blue.
\begin{figure}[htbp] %  figure placement: here, top, bottom, or page
   \centering
   \includegraphics[width=5.5in]{../contour.png} 
   \caption{Red: Contours of the Union3 ``Binned'' model  posterior  $\ln{p_U}$ for a grid of values
   in $w_0$--$w_a$ and several $\Omega_M$.  
   Blue: Contours of  $\ln{p_F}$, which represent the Union3 results replacing the flat prior in node values
   with a flat prior in  $\Omega_M$--$w_0$--$w_a$ space. 
   The stars show the locations of the maxima
   of the two surfaces. Plotted as points are the best fit DESI DESI+Union3
   result and  $\Lambda$CDM.}
   \label{fig:posterior}
\end{figure}


\end{document}  