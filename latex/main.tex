\documentclass[11pt,a4paper]{article}
\pdfoutput=1
\usepackage{jcappub}
\usepackage{natbib}
% \usepackage[T1]{fontenc} % if needed

%\usepackage{geometry}                		% See geometry.pdf to learn the layout options. There are lots.
%\geometry{letterpaper}                   		% ... or a4paper or a5paper or ... 
%\geometry{landscape}                		% Activate for rotated page geometry
%\usepackage[parfill]{parskip}    		% Activate to begin paragraphs with an empty line rather than an indent
%\usepackage{graphicx}				% Use pdf, png, jpg, or eps§ with pdflatex; use eps in DVI mode
								% TeX will automatically convert eps --> pdf in pdflatex		
\usepackage{amssymb, amsmath}

%SetFonts

%SetFonts


\title{Union3 ``Binned''-Model Posterior}
\author[a]{Alex Kim}
\affiliation[a]{Lawrence Berkeley National Laboratory}
\emailAdd{agkim@lbl.gov}
%\date{}							% Activate to display a given date or no date

\abstract{
The Union3 ``Binned''-model posterior has been distributed for third-party cosmology analysis. The posterior prefers a large value of
$\Omega_M$, a small absolute value of $w_0$, and a negative $w_a$, but still accommodates  $\Lambda$CDM; the supernova data alone are not particularly constraining. 
The posterior is calculated for
a  prior that is not flat but rather has non-trivial structure in $\Omega_M$--$w_0$--$w_a$. 
The distributed posterior
and that for a prior that is flat in $\Omega_M$--$w_0$--$w_a$ are shifted relative to each other, but not
at a level that is  statistically significant.   I show how the Union3 posterior can be used
as a prior in a multi-probe fits to different models.
}

\begin{document}
\maketitle

\section{Introduction}
Supernova, BAO, and CMB data are used to fit cosmological and dark-energy parameters in the DESI cosmology
analysis \cite{2024arXiv240403002D}.  The fit to the flat $w_0$--$w_a$ model is discrepant with the $\Lambda$CDM
model at the 3.5$\sigma$ level for the combination of DESI+CMB with Union3  \cite{2023arXiv231112098R}.
Similar level discrepancies are found for the Pantheon+ \cite{2022ApJ...938..113S} and the DES-SN5YR\footnote{Data available at \url{https://github.com/des-science/DES-SN5YR.}} supernova compilations.


%The DESI plus SN results have folks in a tizzy.  Talking with DESI and David it is clear that DESI is incorporating Union3
%into its analyses incorrectly albeit in a subtle way;  it is of interest to explore how misapplication
%of Union3 could affect  DESI $w_0$--$w_a$  fits.

The Union3 ``Binned''\footnote{This name is poorly chosen: there is no binning in this model.} model is designed to compress the  supernova data for convenient use by the community.  
The  model distance modulus
is the sum of the distance modulus of $\Omega_M=0.3$  $\Lambda$CDM plus a second-order spline specified by node values
at a fixed set of 22 redshifts.   Standard normal distributions are used as priors \footnote{Note the difference with the .}.  
The posterior of the nodes is provided  as values of
$\mu = \mu_{\Lambda \text{CDM}}(z;\Omega_M=0.3) + n$ and the Hessian of the $n$ posterior,
where $z$ are the fixed redshifts of the spline and $n$ the node values at the posterior maximum.
In the Union3 paper, cosmological inference is not made from the ``Binned'' model but rather directly from physics-based models.

The ``Binned'' model parameters correspond to values of $\mu$ at the redshift nodes.
Their  posterior can  be used as a prior in joint-probe analyses of models that predict $\mu$ at those redshifts. 
DESI, however, incorporates Union3 data using a different approach that gives rise to inaccuracy in its results.

The objectives of this note are to: Explore what the ``Binned''-model posterior says about $w_0w_a\Lambda$CDM cosmology, which is not normally
fit with  supernova data alone (\S\ref{sec:union3}); Transform the Union3 prior to $\Omega_M$--$w_0$--$w_a$ parameters
(\S\ref{sec:prior});
Examine the new ``Binned'' posterior when switching from the Union3 to DESI prior
(\S\ref{sec:flatposterior}); Show how to include the Union3 ``Binned'' posterior in a joint analysis (\S\ref{sec:joint}).
I conclude with a few conclusions (\S\ref{sec:conclusions}).

\section{Union3 ``Binned'' Model and Posterior}
\label{sec:union3}
Union3 considers several cosmological models, including  flat and open $\Lambda$CDM, $w$CDM, and $w_0w_a$CDM.
In addition to these standard models,  Union3 analyzes a ``Binned'' model where the  distance modulus
is the sum of the distance modulus of $\Omega_M=0.3$  $\Lambda$CDM plus a second-order spline specified by node values, $n$,
at a fixed set of 22 redshifts, $z$.  The distance modulus of this model has significantly more flexibility 
and hence retains more information about the underlying data compared to the physics-motivated models.
The model can be (approximately) viewed as the value of distance modulus at 22 redshifts.
The results of this model fit are now being used by the community.

Union3 uses Bayesian statistics for parameter estimation; as such it requires 
priors on the node values. In the analysis presented in the paper, these priors are taken to be standard normal distributions
$p^\text{Union3}_\text{prior}(n)=  \mathcal{N}(n,1)$.

The physics cosmologies (e.g.\ $w_0w_a$CDM)  are not embedded in the ``Binned'' model but 
they come close; any cosmology's distance moduli at the node redshifts can be exactly replicated by the ``Binned'' model, acknowledging that
there are
inconsistencies between the nodes.  Neglecting this, lets say that the flat $w_0w_a$CDM model
is a subspace of the ``Binned'' model defined by the mapping
\begin{align}
	n &= f(\Omega_M, w_0, w_a; z) \\
	& \equiv \mu_{w_0 w_a \text{CDM}}(z;\Omega_M, w_0, w_a)  - \mu_{\Lambda \text{CDM}}(z;\Omega_M=0.3).
\end{align}

While the Union3 posterior (denoted as $p_U$) is for the 22-dimensional node space, it is of interest to examine the posterior values for the
subspace occupied by the $w_0w_a$CDM
model. 
Values of $\ln{p_U}$ (using the Hessian approximation given by the covariance matrix $C_U$) for a
set of $\Omega_M$--$w_0$--$w_a$ values
 are presented as the red contours in Figure~\ref{fig:posterior}, where the levels are set to 68.27\%, 84.28\%, 91.67\%, and 95.55\% confidence intervals..
The largest $\ln{p_U}$ in $w_0w_a$CDM is at $\Omega_M\approx 0.425$, $w_0 \approx -0.55$, and $w_a \approx -3.7$. $\Lambda$CDM at $\Omega_M \approx 0.35$ is within the 68\% confidence region.
Supernova data alone prefer high $\Omega_M$, a low absolute value of $w_0$, and a negative $w_a$, but do not have
the constraining power to exclude standard cosmology.

\begin{figure}[htbp] %  figure placement: here, top, bottom, or page
   \centering
   \includegraphics[width=5.5in]{../contour.png} 
   \caption{Red: Contours of values of the Union3 posterior $\ln{p_U}+c$ for a
set of $\Omega_M$--$w_0$--$w_a$ values, where $c= 11\ln{(2\pi)} + 0.5\ln({\det{C_U})}$.  The constant $c$ contains the
normalization of the Normal distribution, so $\ln{p_U}+c$  is equivalent to  $-\chi^2/2$.   
   Blue: Contours of  $\ln{p_F} +c  $, which represent the shape of the posterior with a flat DESI prior in  $\Omega_M$--$w_0$--$w_a$ space.   
   The blue contours  also represent the Union3 contribution to the posterior in a joint fit of a flat $w_0w_a$CDM cosmology with DESI priors.
   The maximum of the $\ln{p}_U$  ($\ln{p}_F)$ posterior in this space is shown as the red (blue) star.  (The absolute maxima
   lie outside this space.)
   The contour levels are set to 68.27\%, 84.28\%, 91.67\%, and 95.55\% confidence intervals. 
   Points show the position of DESI's  BAO+CMB+Union3 flat $w_0w_a$ best-fit 
    and  $\Lambda$CDM.}
   \label{fig:posterior}
\end{figure}

The maximum $\chi^2/2$ in the plotted space is $10.33$.  While $w_0w_a\Lambda$CDM is allowed, the best fit ($\chi^2=0$) lies in a
different region of model space. 

\section{Union3 Prior for $n$ in Terms of $\Omega_M$--$w_0$--$w_a$}
\label{sec:prior}
The Union3 prior for the ``Binned'' analysis is designed to keep distance moduli close to a reasonable
fiducial distance modulus function.  It is of interest to see what this prior, which is defined in magnitude
space, looks like in terms of cosmological and dark-energy parameters.

The standard normal prior on the nodes $\mathcal{N}(a,1)$ corresponds to a prior in  $\Omega_M$--$w_0$--$w_a$ of
\begin{align}
p^\text{Union3}_\text{prior}(\Omega_M, w_0,w_a)  & =p^\text{Union3}_\text{prior}(n=f(\Omega_M, w_0, w_a; z))  \sqrt{\det{\left(J^T J\right)}} \\
& = \mathcal{N}(f(\Omega_M, w_0, w_a; z),1)  \sqrt{\det{\left(J^T J\right)}},
\end{align}
where $J$ is the Jacobian matrix of $f$ and $J^TJ$ is the Gram matrix.

\begin{figure}[htbp] %  figure placement: here, top, bottom, or page
   \centering
   \includegraphics[width=5.5in]{../result.png} 
   \caption{Surfaces of the Union3 prior, originally specified in terms of node parameters, transformed to the $w_0w_a$CDM parameterization
   $\ln{p^\text{Union3}_\text{prior}}(\Omega_M, w_0,w_a)$,  for a grid of values
 $w_0$--$w_a$ and $\Omega_M$.   
   Points show the position of the BAO+CMB+Union3 flat $w_0w_a$ best-fit .}
   \label{fig:priors}
\end{figure}

Surfaces of  $\ln{p^\text{Union3}_\text{prior}}(\Omega_M, w_0,w_a)$ values\ for a grid of $\Omega_M$--$w_0$--$w_a$ are shown in
Figure~\ref{fig:priors}.   The prior is not uniform in  $\Omega_M$--$w_0$--$w_a$.
The posterior shown in Figure~\ref{fig:posterior} has a different shape,
demonstrating that it is not being driven by the prior.  Nevertheless,
we will see in the  Section~\ref{sec:flatposterior} that a different reasonable prior can shift the posterior
by an appreciable amount.

%DESI uses the node mean values as the data and the posterior as is error matrix.
%From that perspective, Figure~\ref{fig:posterior} shows the on the $w_0$--$w_a$ manifold
%subspace
%and should not be mistaken as a plot of a posterior.
\section{Posterior for a Flat  $\Omega_M$--$w_0$--$w_a$  Prior}
\label{sec:flatposterior}
Union3 could have used a different prior for the nodes, in particular one
that is flat in $\Omega_M$--$w_0$--$w_a$.
The DESI prior is $\Omega_M \in \mathcal{U}(0.01,0.99)$, $w_0 \in \mathcal{U}(-3,1)$, $w_a \in \mathcal{U}(-3,2)$.
This posterior for this prior is related to the original posterior by
\begin{equation}
p_F(n) = \frac{p_U(n)}{(0.98)(4)(5) \mathcal{N}(n,1)  \sqrt{\det{\left(J^T J\right)}}}.
\label{eq:flatprior}
\end{equation}

For the set of $n$ that are not allowed by $w_0w_a$CDM,  a choice for the prior could be $p_\text{prior}(n)=0$.  The resulting posterior
would effectively be equivalent to that of the Union3 $w_0w_a$CDM model fit.
Preferable is a prior that is continuous over the full space and of more general applicability,
perhaps through analytic continuation of $\sqrt{\det{\left(J^T J\right)}}$ into the full space.
The specific choice is not important to this note as long as Eq.~\ref{eq:flatprior} holds.

Values of $\ln{p}_F$ are shown for  $\Omega_M$--$w_0$--$w_a$ as blue contours in Figure~\ref{fig:posterior}.\footnote{Keep in mind that the Gaussian approximation of $p_U$ may not characterize the true posterior at all points of
phase space including the most probable value of $p_F$.}
The largest $\ln{p_U}$ in $w_0w_a$CDM is a $\Omega_M\approx 0.475$, $w_0 \approx -0.425$, and $w_a \approx -5.9$. $\Lambda$CDM 
is within the 68\% confidence region.  This is a shift from the original Union3 posterior, though both maxima are well within their counterpart
posterior's 68\% confidence regions.  An interesting feature is that the two posteriors have nearly the same values for $w_0$--$w_a$ when $\Omega_M=0.3$, the fiducial cosmology of the ``Binned'' model.

\section{Union3 in Joint Analyses}
\label{sec:joint}
Union3 results can be combined with other datasets in multi-probe analyses.
Consider independent data $d$ used to fit a model parameterized by $\theta$
that predicts distance moduli at the ``Binned'' model redshifts as $n=f(\theta)$.
The Union3 posterior can be used as a prior for $n$.  The posterior of the joint analysis is
\begin{align}
p(\theta|d) &\propto \int  p(d|n,\theta)p(\theta|n)p_U(n) dn\\
&=  p(d|f(\theta),\theta) \frac{p_U(f(\theta))} {\sqrt{\det{J^TJ}}}, \label{eq:correct}
\end{align}
where $J$ is the Jacobian of $f$.
Implicit in the above is the Union3 prior
that went into calculating $p_U$; other priors can be considered by factoring in their ratios with respect to the 
original prior.

Consider a joint fit of the  $w_0w_a$CDM model with DESI priors.  The Union3 contribution to the posterior turns out to be
what has already been presented in 
Equation~\ref{eq:flatprior} and plotted as the blue contours in
Figure~\ref{fig:posterior} 

The DESI paper takes a different approach.  The Union3 posterior best-fit parameters are thought of as estimators and used as
 data.  The posterior is
\begin{align}
p(\theta | d, \hat{n}) & \propto p(d| \theta) p( \hat{n}| \theta) p^\text{DESI}_\text{prior}(\theta) \\
& \approx p(d| \theta) p_U(\hat{n})  p^\text{DESI}_\text{prior}(\theta)  \\
& \approx p(d| \theta) \mathcal{N}( \hat{n}-f(\theta) , C_U)  p^\text{DESI}_\text{prior}(\theta) \label{eq:DESI}.
\end{align}
The  likelihood
 term $ p( \hat{n}| \theta)$ is non-analytic and non-trivial to map
(see \cite{2013ApJ...764..116W} for a worked example of a likeliood-free supernova-cosmology analysis).
Approximating the likelihood
by $\mathcal{N}( \hat{n}-f(\theta) , C_U)$ leads to an inaccurate determination
of the posterior, by an amount that I don't attempt to calculate in this note.
Nevertheless, the posterior given in Equation~\ref{eq:correct} is related to the reported DESI result Equation~\ref{eq:DESI}
by a shift from the red to blue contours in Figure~\ref{fig:posterior}, which would induce a slight shift away from $\Lambda$CDM
and toward higher $\Omega_M$.
The current DESI implementation in Cobaya can be changed by swapping
\begin{equation}
 \mathcal{N}( \hat{n}-f(\theta) , C_U) = -\frac{1}{2} (f(\Omega_m, w_0, w_a) - \hat{n})^T C_U^{-1} (f(\Omega_m, w_0, w_a) - \hat{n})
\end{equation}
with
\begin{equation}
 \frac{p_U(f(\theta))} {p^\text{Union3}_\text{prior}(\hat{n}) \sqrt{\det{J^TJ}}} =\frac{ -\frac{1}{2} f(\Omega_m, w_0, w_a) ^T C_U^{-1}  f(\Omega_m, w_0, w_a) }
 {\mathcal{N}(\hat{n},1) \sqrt{\det{J^TJ}}}.
\end{equation}
It is eminently simpler and safer to treat the Union3 parameters and their posterior as such, and not as data.


Caution should be exercised when multi-probe analyses favor regions far from the Union3 best fit  where the Hessian approximation of the
posterior may be less secure, or in regions where $\sqrt{\det{J^TJ}}$ has structure.  For example, while the plotted contours in Figure~\ref{fig:priors}
appear to be consistent with a Gaussian surface, this does not persist at lower contour levels even at high $w_a$ within the ``Binned'' posterior 68\% confidence region.

\section{Conclusions}
\label{sec:conclusions}
Multi-probe  analysis is required to get the most cosmological information out of the plethora of available data. 
Compressed data products from different experimental groups can be combined to achieve this purpose.
Although not as optimal as simultaneously fitting all data, this approach simplifies algorithms and allows a larger
fraction of
the community to work with the data.

As such, Union3 has released a posterior for a general model that summarizes the Hubble diagram inferred from its data.
The model is parameterized by distance moduli on a grid of redshifts and is (approximately) a superspace that
contains a broad range of models.     The posterior of any embedded model can be
extracted from the  ``Binned''-model posterior and used as a prior in subsequent analyses.

Despite its simplicity, it is important for users to understand the assumptions made in the model and the priors used
in generating its parameter posterior.  Then they can properly incorporate Union3 into their own analyses.

Establishing a standard flexible model and prior though which all experiments can 
summarize their results would facilitate proper usage by the community.

\bibliographystyle{JHEP}
\bibliography{apj-jour,ref}


\end{document}  