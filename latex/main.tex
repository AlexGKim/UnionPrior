\documentclass[11pt, oneside]{article}   	% use "amsart" instead of "article" for AMSLaTeX format
\usepackage{geometry}                		% See geometry.pdf to learn the layout options. There are lots.
\geometry{letterpaper}                   		% ... or a4paper or a5paper or ... 
%\geometry{landscape}                		% Activate for rotated page geometry
%\usepackage[parfill]{parskip}    		% Activate to begin paragraphs with an empty line rather than an indent
\usepackage{graphicx}				% Use pdf, png, jpg, or eps§ with pdflatex; use eps in DVI mode
								% TeX will automatically convert eps --> pdf in pdflatex		
\usepackage{amssymb, amsmath}

%SetFonts

%SetFonts


\title{Unity ``Binned'' Prior}
\author{Alex Kim}
%\date{}							% Activate to display a given date or no date

\begin{document}
\maketitle

The DESI plus SN results have folks in a tizzy.  Talking with DESI and David it is clear that DESI is incorporating Union3
into its analyses incorrectly albeit in a subtle way;  it is of interest to explore how misapplication
of Union3 could affect  DESI $w_0$--$w_a$  fits.

In the Unity ``Binned'' model, the distance modulus
is the sum of the distance modulus of $\Omega_M=0.3$  $\Lambda$CDM plus a second-order spline specified by node values
at a fixed set of 22 redshifts.  The priors on the node values are taken to be standard normal distributions\footnotemark[1].  
The posterior of the nodes is provided by David to DESI as values of
$\mu_{\Lambda \text{CDM}}(z;\Omega_M=0.3) + n$ and the Hessian of the $n$ posterior,
where $z$ are the fixed redshifts of the spline and $n$ the node values at the posterior maximum.
Note the misnomer, there is no binning in this model.

\footnotetext[1]{The Union3 preprint states that the prior is flat.  However David has found looking back at his code
that the prior is standard normal.}

The $w_0$--$w_a$ cosmology is not embedded in the `Binned'' model but 
it comes close; each $w_0$--$w_a$ luminosity distance is almost replicated
by the `Binned'' model with node values that match the predictions of $w_0$--$w_a$  at the node redshifts, acknowledging that
there are
inconsistencies between the physics and spline predictions between the nodes.  For the purposes of this work, I consider
as an alternative  ``$w_0$--$w_a$'' model  the subspace/manifold of node values allowed by the true $w_0$--$w_a$
model.  The transformation between the subspace parameters $\{\Omega_M, w_0, w_a\}$ and their corresponding
node-value $n$ parameters is given by
\begin{align}
	n &= f(\Omega_M, w_0, w_a; z) \\
	& \equiv \mu_{w_0 w_a \text{CDM}}(z;\Omega_M, w_0, w_a)  - \mu_{\Lambda \text{CDM}}(z;\Omega_M=0.3).
\end{align}

The standard normal prior on the nodes $\mathcal{N}(a,1)$ corresponds to a prior on the  $w_0$--$w_a$ manifold
\begin{equation}
p_U(\Omega_M, w_0,w_a) = \mathcal{N}(f(\Omega_M, w_0, w_a; z),1)  \sqrt{\det{\left(J^T J\right)}},
\end{equation}
where $J$ is the Jacobian matrix of $f$ and $J^TJ$ is the Gram matrix.
Surfaces of $\ln{p_U}(\Omega_M, w_0,w_a)$ values\ for a grid of values of $\Omega_M$
are plotted in Figure~\ref{fig:priors}.  This prior is not flat in  $\Omega_M, w_0, w_a$-space.

\begin{figure}[htbp] %  figure placement: here, top, bottom, or page
   \centering
   \includegraphics[width=5.5in]{../result.png} 
   \caption{Surfaces of $\ln{p_U}(\Omega_M, w_0,w_a)$  for a grid of values
 $w_0$--$w_a$ and $\Omega_M$.    The star (in the upper-left plot) shows the location of the maximum.
   Points show the position of the best fit  DESI+Union3
   result and  $\Lambda$CDM.}
   \label{fig:priors}
\end{figure}

Values of the Union3 posterior (using the Hessian approximation) for a
set of $\Omega_M$--$w_0$--$w_a$ values
are shown as the red contours in Figure~\ref{fig:posterior}.\footnotemark[2]
This posterior, denoted as $p_U$, has the standard-normal prior baked in. 
The largest $\ln{p_U}$ on the manifold (shown as a red star) is at the high end of $\Omega_M$ and the high--low
part of $w_0$--$w_a$ space.  
Supernova data alone do not significantly constrain the $w_0$--$w_a$ (the plotted contours do not correspond
to standard confidence intervals).

\footnotetext[2]{Keep in mind that the Gaussian approximation of $p_U$ may not characterize the true posterior at all points of
phase space including the most probable value of $p_F$.}

%DESI uses the node mean values as the data and the posterior as is error matrix.
%From that perspective, Figure~\ref{fig:posterior} shows the on the $w_0$--$w_a$ manifold
%subspace
%and should not be mistaken as a plot of a posterior.

Union3 could have used a different prior for the nodes.  Of interest is a
prior that is flat in $\Omega_M$--$w_0$--$w_a$ on the manifold.
The corresponding posterior is related to the original posterior by
\begin{equation}
p_F(n) \propto \frac{p_U(n)}{\mathcal{N}(n,1)  \sqrt{\det{\left(J^T J\right)}}}.
\end{equation}
I haven't specified what the prior is off of the manifold.  One choice is $p(n)=0$ under the assumption
of a $w_0$--$w_a$ cosmology.  The resulting posterior, however, would not easily be incorporated into the DESI analysis
and useful information encoded in the 22-dimensional parameters would be lost.
Preferable is a prior that is continuous over the full space and hence more applicable for DESI,
perhaps through analytic continuation of $\sqrt{\det{\left(J^T J\right)}}$ into the larger space.


Values of the posterior $p_F$ assuming a flat
prior in $\Omega_M$--$w_0$--$w_a$  are shown as blue contours in Figure~\ref{fig:posterior}.\footnotemark[2]
The largest value of $\ln{p}_F$ (shown as a blue star) is on the low end of $\Omega_M$ and the  low--high
part of $w_0$--$w_a$ space.  
The
change in prior does not result in a statistically important change in the $\ln{p}-\ln{p}_\text{max}$ values on the manifold;
Union3 SN data alone do not strongly constrain the 22-dimensional node space.
Nevertheless, we see that a change in prior can result in a large shift in the posterior.


\begin{figure}[htbp] %  figure placement: here, top, bottom, or page
   \centering
   \includegraphics[width=5.5in]{../contour.png} 
   \caption{Red: Contour plot of values of the Union3 posterior $\ln{p}_U(f(\Omega_M, w_0, w_a; z))$ (using the Hessian approximation) for a
set of $\Omega_M$--$w_0$--$w_a$ values.
   Blue: Contours of  $\ln{p_F}(f(\Omega_M, w_0, w_a; z))$ values, which represent a new Union3 posterior with a flat prior in  $\Omega_M$--$w_0$--$w_a$ space.   
   The maximum of the $\ln{p}_U$  ($\ln{p}_F)$ posterior within the manifold is shown as the red (blue) star.
   The contour levels are set to $\Delta \ln{p}=1.76$ (corresponding to a 68\% confidence region on a 3-dimensional space), much smaller than the $\Delta \ln{p}=12.29$ for the 68\% confidence region in the 22-dimensional
   space for which the posterior applies. 
   These plots should not be mistaken as posterior plots in  $\Omega_M, w_0, w_a$. 
   Points show the position of the best fit DESI DESI+Union3
   result and  $\Lambda$CDM.}
   \label{fig:posterior}
\end{figure}

DESI takes the Union3 ``Binned"-model parameter posterior output as data, interprets the covariance matrix as the Gaussian uncertainty,
and includes them in a larger likelihood  to fit the $w_0$--$w_a$ flat cosmology model.   It thus seems appropriate
to use a Union3 $p_F$ rather than $p_U$ posterior, which would (probably) shift DESI+Union3 contours toward $\Lambda$CDM.
Though the shift may be statistically insignificant when inferring $w_0$--$w_a$ with SN data alone it could have an important effect in the joint analysis,
where the larger parameter space plays a non-trivial role in breaking degeneracies.

What is the solution? Of course the best thing would be to transport
the supernova-cosmology likelihood into the joint analysis.  This may not be a crazy idea, the likelihood is well documented
and I have seen the David's code and it is fairly straightforward.  David uses STAN  (HMC
with no U-Turn sampling) to run the MCMC.
Short of that, we should ask
David to run Union3 with a new prior considering that the Gaussian approximation may not accurately describe the provided posterior, 
What the new prior should be is something to think about.  
\end{document}  